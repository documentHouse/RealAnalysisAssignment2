%\documentclass[11pt,reqno]{amsart}
\documentclass[11pt,reqno]{article}
\usepackage[margin=.8in, paperwidth=8.5in, paperheight=11in]{geometry}
%\usepackage{geometry}                % See geometry.pdf to learn the layout options. There are lots.
%\geometry{letterpaper}                   % ... or a4paper or a5paper or ... 
%\geometry{landscape}                % Activate for for rotated page geometry
%\usepackage[parfill]{parskip}    % Activate to begin paragraphs with an empty line rather than an indent7
\usepackage{graphicx}
\usepackage{pstricks}
\usepackage{amssymb}
\usepackage{epstopdf}
\usepackage{amsmath}
\usepackage{subfigure}
\usepackage{caption}
\pagestyle{plain}
%\renewcommand{\topfraction}{0.3}
%\renewcommand{\bottomfraction}{0.8}
%\renewcommand{\textfraction}{0.07}
\DeclareGraphicsRule{.tif}{png}{.png}{`convert #1 `dirname #1`/`basename #1 .tif`.png}

\title{Real Analysis $\mathbb{I}$: \\ Assignment 2}
\author{Andrew Rickert}
\date{Started: January 2, 2011 \\ \hspace{1pt} Ended: January ??, 2010}                                           % Activate to display a given date or no date

\begin{document}
\maketitle


% Page 1
\begin{flushleft} 
\textbf{Class 18.100B} - Problem 1\\
\rule{500pt}{1pt}\\
\end{flushleft} 

We would like to show that the sentence $'\emptyset \subset A'$ is true where $A$ is any set. This is equivalent to the assertion that the empty set is the subset of any set. To this end we are to show that $x \in \emptyset$ implies $x \in A$. We know however that $x \in \emptyset$ is false for any $x$ therefore the sentence  $x \in \emptyset \implies x \in A$ is trivially true. This means that the sentence  $'\emptyset \subset A'$ is true.


\vspace{15pt}
\begin{flushleft} 
\textbf{Class 18.100B} - Problem 2\\
\rule{500pt}{1pt}\\
\end{flushleft} 

To show that $||x| - |y|| \le |x - y|$ we need to show that both $|x| - |y| \le |x - y|$ and \\$-(|x| - |y|) = |y| - |x| \le |x - y|$ from the definition of the absolute value. We show the first as follows

\begin{eqnarray*}
|x| &=& |x - y + y| \\
|x| &\le& |x - y| + |y| \quad \text{by the triangle inequality}
\end{eqnarray*}
so we have $|x| \le |x - y| + |y|$ which implies $|x| -|y| \le |x - y| $ which gives the first part of the theorem. Therefore we calculate as follows:

For the second part we note that $|x| = |-x|$ from the definition of absolute value.

\begin{eqnarray*}
|y| &=& |y - x + x| \\
|y| &\le& |y - x| + |x| \quad \text{by the triangle inequality} \\
|y| &\le& |-(x - y)| + |x| \\
|y| &\le& |x - y| + |x| \quad \text{from the definition of absolute value}
\end{eqnarray*}
so we have $|y| \le |x - y| + |x|$ which implies $|y| -|x|  = -(|x|-|y|) \le |x - y| $ which gives the second part of the theorem. 

\newpage
\vspace{15pt}
\begin{flushleft} 
\textbf{Class 18.100B} - Problem 3\\
\rule{500pt}{1pt}\\
\end{flushleft} 

\noindent Part a) $M = \lbrace \frac{|x|}{1 + |x|} : x \in \mathbb{R} \rbrace $
\vspace{10pt}

We start by noting that $0 \le |x|$ also $|x| < |x| + 1$ so $0 \le |x| < |x| + 1 \implies 0 \le \frac{|x|}{1+ |x|}$. Since when $x = 0$ we also have $ \frac{|x|}{1+ |x|} = 0$ then inf $M$ = 0. 

For the supremum we note that $|x| < |x| + 1 \implies \frac{|x|}{|x|+1} < 1$ that is, $M$ is bounded by 1. For any $\epsilon$ there exists $|x|$ such that $0 < \frac{1}{\epsilon} < |x|$ this implies that $\frac{1}{\epsilon} < |x| + 1$ so $\frac{1}{|x| + 1} < \epsilon$ but then $1 - \frac{1}{|x| + 1}  > 1 - \epsilon$ so we have $\frac{|x|}{|x| + 1}  > 1 - \epsilon$, which is to say any real less than 1 is not an upper bound for $M$. This means that sup $M$ = 1.

\vspace{10pt}
\noindent Part b) $M = \lbrace \frac{x}{1 + x} : x > -1 \rbrace $
\vspace{10pt}

\noindent Part c) $M = \lbrace x + \frac{1}{x} : \frac{1}{2} < x < 2 \rbrace $
\vspace{10pt}

\vspace{15pt}
\begin{flushleft} 
\textbf{Class 18.100B} - Problem 4\\
\rule{500pt}{1pt}\\
\end{flushleft} 

We can take the set $S = A \cap B \cap C$ to mean $S$ the set which has the properties of all three sets. First, this means that $S$ will only contain positive integers less or equal to 200. Second, of those 200 integers we only pick those that are expressed as a product of exactly three primes. Finally, because we require that the integer is not divisible by a square then none of the three prime numbers can be equal. Our set $S$ is then the collection of integers less than 200 which are expressed as the product of three different primes.

The lowest two primes are 2 and 3, therefore the highest prime that we can use to produce an integer less than 200 is 31. We then proceed by picking the next lowest prime and finding all the distinct prime triples that keep the product under 200. This is easy since we only check (2,3,29) and (2,5,29) to see that (2,3,29) is the only possible prime triple that has a produce less than 200 . We proceed in this manner incrementally checking the small set of possible triples to produce the following set of values.
\vspace{10pt}

\begin{tabular}{c c c}
(2,3,31) & (2,3,11) & (2,5,7)\\
(2,3,29) & (2,5,11) & (3,5,7)\\
(2,3,23) & (3,5,11)& (2,3,5)\\
(2,3,19) & (2,7,11) & \\
(2,5,19) & (2,3,9) & \\
(2,3,17) & (2,5,9) & \\
(2,5,17) & (3,5,9) & \\
(2,3,13) & (2,7,9) & \\
(2,5,13) & (3,7,9) & \\
(3,5,13) & (2,3,7) & \\
\end{tabular}

\vspace{10pt}
Prime decompositions are unique which allows us to translate the table above into an explicit definition of S:
\[
S = \{ 30,42,54,66, 70,78,90,102,105,110,114,126,130,135,138,154,165,170,174,186,189,190,195 \}
\]

\vspace{15pt}
\begin{flushleft} 
\textbf{Class 18.100B} - Problem 5\\
\rule{500pt}{1pt}\\
\end{flushleft} 

\vspace{15pt}
\begin{flushleft} 
\textbf{Class 18.100B} - Problem 6\\
\rule{500pt}{1pt}\\
\end{flushleft} 


\vspace{15pt}
\begin{flushleft} 
\textbf{Class 18.100B} - Problem 7\\
\rule{500pt}{1pt}\\
\end{flushleft} 

\end{document}  