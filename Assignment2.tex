%\documentclass[11pt,reqno]{amsart}
\documentclass[11pt,reqno]{article}
\usepackage[margin=.8in, paperwidth=8.5in, paperheight=11in]{geometry}
%\usepackage{geometry}                % See geometry.pdf to learn the layout options. There are lots.
%\geometry{letterpaper}                   % ... or a4paper or a5paper or ... 
%\geometry{landscape}                % Activate for for rotated page geometry
%\usepackage[parfill]{parskip}    % Activate to begin paragraphs with an empty line rather than an indent7
\usepackage{graphicx}
\usepackage{pstricks}
\usepackage{amssymb}
\usepackage{epstopdf}
\usepackage{amsmath}
\usepackage{subfigure}
\usepackage{caption}
\pagestyle{plain}
%\renewcommand{\topfraction}{0.3}
%\renewcommand{\bottomfraction}{0.8}
%\renewcommand{\textfraction}{0.07}
\DeclareGraphicsRule{.tif}{png}{.png}{`convert #1 `dirname #1`/`basename #1 .tif`.png}

\title{Real Analysis $\mathbb{I}$: \\ Assignment 2}
\author{Andrew Rickert}
\date{Started: January 2, 2011 \\ \hspace{1pt} Ended: January ??, 2010}                                           % Activate to display a given date or no date

\begin{document}
\maketitle


% Page 1
\begin{flushleft} 
\textbf{Class 18.100B} - Problem 1\\
\rule{500pt}{1pt}\\
\end{flushleft} 

We would like to show that the sentence $'\emptyset \subset A'$ is true where $A$ is any set. This is equivalent to the assertion that the empty set is the subset of any set. To this end we are to show that $x \in \emptyset$ implies $x \in A$. We know however that $x \in \emptyset$ is false for any $x$ therefore the sentence  $x \in \emptyset \implies x \in A$ is trivially true. This means that the sentence  $'\emptyset \subset A'$ is true.


\vspace{15pt}
\begin{flushleft} 
\textbf{Class 18.100B} - Problem 2\\
\rule{500pt}{1pt}\\
\end{flushleft} 

To show that $||x| - |y|| \le |x - y|$ we need to show that both $|x| - |y| \le |x - y|$ and \\$-(|x| - |y|) = |y| - |x| \le |x - y|$ from the definition of the absolute value. We show the first as follows

\begin{eqnarray*}
|x| &=& |x - y + y| \\
|x| &\le& |x - y| + |y| \quad \text{by the triangle inequality}
\end{eqnarray*}
so we have $|x| \le |x - y| + |y|$ which implies $|x| -|y| \le |x - y| $ which gives the first part of the theorem. Therefore we calculate as follows:

For the second part we note that $|x| = |-x|$ from the definition of absolute value.

\begin{eqnarray*}
|y| &=& |y - x + x| \\
|y| &\le& |y - x| + |x| \quad \text{by the triangle inequality} \\
|y| &\le& |-(x - y)| + |x| \\
|y| &\le& |x - y| + |x| \quad \text{from the definition of absolute value}
\end{eqnarray*}
so we have $|y| \le |x - y| + |x|$ which implies $|y| -|x|  = -(|x|-|y|) \le |x - y| $ which gives the second part of the theorem. 

\vspace{15pt}
\begin{flushleft} 
\textbf{Class 18.100B} - Problem 3\\
\rule{500pt}{1pt}\\
\end{flushleft} 

\vspace{15pt}
\begin{flushleft} 
\textbf{Class 18.100B} - Problem 4\\
\rule{500pt}{1pt}\\
\end{flushleft} 

\vspace{15pt}
\begin{flushleft} 
\textbf{Class 18.100B} - Problem 5\\
\rule{500pt}{1pt}\\
\end{flushleft} 

\vspace{15pt}
\begin{flushleft} 
\textbf{Class 18.100B} - Problem 6\\
\rule{500pt}{1pt}\\
\end{flushleft} 


\vspace{15pt}
\begin{flushleft} 
\textbf{Class 18.100B} - Problem 7\\
\rule{500pt}{1pt}\\
\end{flushleft} 

\end{document}  